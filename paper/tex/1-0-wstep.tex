\newpage % Rozdziały zaczynamy od nowej strony.
\section{Wstęp}
Kwestia wiarygodności informacji napływających do nas ze wszystkich mediów stała się ostatnio gorącym tematem. Mimo faktu, że problem występuje od zawsze, w\,ostatnich latach zwrócił na siebie dużo uwagi. Przy ilości informacji jaka przepływa przez Internet, w\,szczególności dzięki możliwości udostępniania wszystkiego poprzez portale społecznościowe, szerzenie dezinformacji w\,formie postów stało się bardzo proste i\,szybkie. Mogą one przyjmować rozmaitą formę zaczynając na krótkiej formie tekstowej jaką są Tweety, poprzez kłamliwe artykuły naukowe, przerobione zdjęcia na zmanipulowanych filmach wideo kończąc. Należy więc podjąć próbę wykrycia prawdziwości pojawiających się treści. Można to wykonać przez pracę człowieka lub automatycznie.
\par
W ostatnich latach powstaje coraz więcej inicjatyw próbujących zrozumieć i\,powstrzy- mać to zjawisko skupiając zarówno się na podłożu socjologicznym jak również tworząc nowe rozwiązania informatyczne których zadaniem będzie pomoc użytkownikom w\,rozróżnieniu prawdy od dezinformacji. Takich projektów jednak jest nadal bardzo mało i\,w\,większości nie obejmują one innych języków niż angielski.
\par
W poniższej pracy przedstawiono opis problemu jakim są nierzetelne informacje oraz zebrano dotychczasowe rozwiązania oceny prawdziwości treści. Następnie opracowano badania porównujące cechy kont oraz publikowanych treści w\,mediach społecznościowych. Badano treści w\,języku polskim oraz porównywano konta zakwalifikowane jako nierzetelne do kont określanych jako wiarygodne. Takiego podziału dokonano na podstawie kilku niezależnych źródeł. Na podstawie zebranych danych dokonano badania automatycznej klasyfikacji treści pochodzących z\,portalu społecznościowego na podstawie ich kontekstu. Klasyfikowano czy post należy do klasy nierzetelnych czy też nie wykorzystując wiedzę jego odbiorcach.
\par
Na wstępie przeprowadzono rozpoznanie problemu, czynniki mu sprzyjające oraz jego wpływ na rzeczywistość. Następnie na podstawie dostępnej literatury i\,istniejących badań przedstawiono najbardziej popularne sposoby rozwiązania problemu klasyfikacji treści pod względem ich wiarygodności. 
Okazuje się, że większość istniejących projektów w\,tym temacie skupia się wyłącznie na tematyce polityki w\,USA. Jedynie pojedyncze zespoły podjęły do tej pory działania obejmujące obszary związane z\,innymi językami niż angielski oraz inną strefą tematyczną. Według najlepszej wiedzy autora, nie istnieją w\,tym momencie żadne inicjatywy podejmujące temat klasyfikacji informacji w\,mediach społecznościowych w\,języku polskim. Z tego powodu poniższa praca w\,dalszej swojej części skupia się na rozpoznaniu problemu wiarygodności informacji w\,polskiej przestrzeni internetowej. Przytoczone zostaną badania opinii społecznej oraz zanalizowane zostaną dostępne badania obejmujące tematy Unii Europejskiej. 
\par
W drugiej części pracy zostanie przeprowadzona oraz opisana analiza wybranych danych pochodzących z\,polskojęzycznych postów opublikowanych na mediach społecznościowych. Celem tej analizy będzie znalezienie właściwości i\,charakterystyk kont uznanych za nierzetelne. W\,celu analizy porównawczej konta te zostaną przeciwstawione kontom należącym do profesjonalnych serwisów informacyjnych, serwisów o dużej liczbie odsłon oraz portali zajmujących się wyłącznie sprawdzaniem prawdziwości innych mediów (tzw.\,factchecking).
\par
Ostatnim badaniem przeprowadzonym w\,ramach tej pracy będzie dokonanie próby automatycznej klasyfikacji treści pochodzących z\,mediów społecznościowych na podstawie ich kontekstu. Kontekstem, który zostanie przyjęty będzie interakcja użytkowników portalu z\,badanymi treściami. Celem tego badania jest sprawdzenie możliwości wykonania takiej klasyfikacji i\,jej precyzji. Dodatkowo zostanie zbadane jak duża musi być zaklasyfikowana próbka treningowa, aby otrzymać zadowalające wyniki klasyfikacji. Do wykonania tej klasyfikacji zostaną użyte dwa algorytmy. Pierwszym z\,nich jest klasyczny algorytm uczenia maszynowego jakim jest regresja logistyczna, a\,drugim będzie zastosowanie algorytmu etykietowania przez użytkowników używając właściwości rozprzestrzeniania wiedzy w\,sieci. 
\par
Na koniec podsumowana zostanie cała zdobyta wiedza oraz wyniki przeprowadzonych badań. Zostaną zebrane wnioski oraz wynikająca z\,nich nauka dla przyszłych inicjatyw chcących badać lub automatycznie wykrywać nieprawdziwe informacje publikowane za pomocą mediów społecznościowych. 

