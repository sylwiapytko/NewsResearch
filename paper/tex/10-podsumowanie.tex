\newpage % Rozdziały zaczynamy od nowej strony.
\section{Podsumowanie}
W powyższej pracy przedstawiono problem jakim są nierzetelne informacje skupiając się na tych występujących w Internecie a w szczególności w mediach społecznościowych. Jako nierzetelne informacje przyjęto te, co do których istnieje wystarczająco wysokie prawdopodobieństwo, że nie zawierają one prawdy lub są intencjonalnie dezinformacyjne. Są one przeciwstawiane informacjom wiarygodnym, czyli takim co do których zakładamy, że są prawdziwe. Dokonywanie decyzji o prawdziwości lub nierzetelności informacji jest bardzo trudnym zadaniem. Należy bowiem przyjąć, że pewne fakty są bezsprzecznie prawdziwe oraz znaleźć źródła, które są uznane za wiarygodne. Taka decyzja jednak nigdy nie będzie całkowicie obiektywna i niepodważalna. 
\par
Z tego też powodu na początku pracy omówiono słownikowe definicje dezinformacji oraz jego synonimu fake news. Następnie omówiono problematykę nierzetelnych informacji skupiając się na ich występowaniu w mediach społecznościowych. Omówiono podłoże tego problemu, czynniki mu sprzyjające oraz konsekwencje jakie mogą wypływać z jego rozpowszechniania się. Wzięto pod uwagę również omówienie czynników psychologicznych i społecznych, które silnie oddziałują na użytkowników internetowych sieci społecznych. 
\par 
Następnie na podstawie dostępnej literatury zostały przedstawione i porównane najbardziej powszechne metody na klasyfikację informacji pod względem ich prawdziwości. Skupiono się na informacjach w formie tekstowej, ale do nich nie ograniczono. Metody podzielono na dwie główne kategorie: manualną oraz automatyczną. Metoda manualna to taka, która jest zależna od pracy człowieka i dzieli się pod względem jakości pojedynczej opinii i liczby zebranych opinii. Pierwszą z nich jest praca manualna ekspertów tematu, natomiast druga polega na zebraniu bardzo dużej liczby danych od większej liczby osób. Ta pierwsza metoda jest najczęściej wykorzystywana w serwisach specjalizujących się w ocenianiu informacji, natomiast drugą używa się jako metodę zebrania danych do wykonania klasyfikacji automatycznej. Tą z kolei można podzielić pod względem metody i\,typu danych jakie wykorzystuje do etykietowania. Można bowiem wykorzystać wiedzę semantyczną o treści informacji używając metod NLP, metod sieci wiedzy z wykorzystaniem np. Wikipedii lub skupić się nie na treści informacji ale na jej kontekście. Ostatnia metoda jest najczęściej używana do klasyfikacji informacji pochodzących z mediów społecznościowych dzięki ich charakterystyce interakcji twórców z użytkownikami. 
\par
W kolejnym rozdziale przedstawiono badania opinii publicznej na temat wiarygodności informacji w mediach. Przedstawiono wyniki najciekawszych badań obejmujących tematykę polskich mediów oraz polskich internautów. Następnie omówiono dokładniej istniejące badania podejmujące temat prób automatycznej klasyfikacji informacji związanych z Polską oraz Unią Europejską.  Na podstawie badań zespołu z Uniwersytetu w Oxfordzie okazuje się bowiem, że nawet 20\% badanych tweetów może pochodzić ze źródeł uznane jako nierzetelne.  Przedstawiono również badania wpływu botów na polskojęzyczną przestrzeń na platformie Twitter. 
\par
W drugiej części pracy przedstawiono i omówiono stworzony projekt mający na celu analizę danych pochodzących z platformy Twitter z podziałem tych danych na klasy pod względem ich rzetelności.  Ze względu na brak istniejącej klasyfikacji pojedynczych informacji, zdecydowano się  na podział całych kont. Takie podejście było też zastosowane w innych pracach zajmujących się podobnymi tematami. Aby uzyskać taką klasyfikację oparto się na kilku różnych źródłach, w głównej mierze na badaniach zespołu COMPROP z Oxfordu. Konta serwisów uznanych za nierzetelne przeciwstawiono kontom serwisów profesjonalnych i o dużej popularności liczonej w liczbie odsłon. Tą drugą klasę nazwano wykorzystując angielskie słowo mainstream. Dodatkowo również zbadano konta należące do serwisów specjalizujących się w sprawdzaniu prawdziwości informacji czyli tzw. factcheck. 
\par
Z przeprowadzonej analizy wyciągnięto wnioski dotyczące charakterystyk tych grup. Odkryto, że konta z największą liczbą odbiorców czyli followersów otrzymują średnio więcej udostępnień swoich postów ale nie jest to jednoznaczne. Największą popularność, w znaczeniu liczby udostępnień, posiadają posty publikowane przez konta typu factcheck, mimo że te mają stosunkowo mało followersów. Popularność tweetów zależy bardziej od konkretnego konta i nie ma odzwierciedlenia w popularności klasy do której należy. Chciano zbadać również relacje pomiędzy kontami oraz zewnętrznymi domenami które konta udostępniają. Okazuje się jednak, że udostępnienia informacji ograniczają się do udostępnień swojej grupy wydawniczej. Najbardziej obiecującą charakterystyką kont są użytkownicy którzy udostępniają ich posty. Mimo, że około połowa zebranych użytkowników udostępniła tylko dwa posty to istnieje duża grupa użytkowników, która udostępnia ich dużo więcej. Dodatkowo część użytkowników jest spolaryzowana na konkretny typ informacji. Najbardziej aktywnymi użytkownikami są ci udostępniający posty pochodzące z kont oznaczonych jako nierzetelne. Na podstawie udostępnień użytkowników można określić prawdopodobieństwo ich preferencji do jednej z klas.
\par
Ten fakt wykorzystano przy tworzeniu automatycznej klasyfikacji informacji pochodzących z Twittera. Swoje badania oparto na istniejącej pracy w której porównano dwie metody klasyfikacji danych pochodzących z platformy, Facebook która uzyskała bardzo wysokie wyniki precyzji klasyfikacji. Jako cel badań postawiono sprawdzenie precyzji klasyfikacji z wykorzystaniem kontekstu jakim są udostępnienia postów przez użytkowników. Dodatkowym celem było sprawdzenie jak duży musi być zbiór zawierający etykiety w stosunku do zbioru testowego aby otrzymać zadowalającą precyzję. Okazało się że wystarczy aby jedynie 10\% postów znajdywało się w zbiorze uczącym aby otrzymać precyzję na poziomie 90\%.  Takie wyniki uzyskano zarówno dla regresji logistycznej jak również dla drugiej metody jaką było przedstawienie danych w formie grafu, gdzie wierzchołkami są zarówno posty jak i użytkownicy.
\par 
Przeprowadzono również test sprawdzający jak dobra jest transmisja wiedzy o przynależności badanych kont do klas gdy ze zbioru uczącego wyłączymy część z nich. Te wyniki okazały się być niewiele lepsze od losowego zgadywania. Powodem tego prawdopodobnie jest fakt, że niektóre z kont nie miały wystarczająco wspólnych użytkowników z\,innymi z\,kont.
\par
Podsumowując treści przedstawione w pracy, przedstawiono i opisano problem nierzetelnych informacji skupiając się na tych występujących w mediach społecznościowych pod postacią postów. Przedstawiono powody występowania dezinformacji oraz metody dokonujące ich klasyfikacji. Następnie przeprowadzono badania na temat nierzetelnych informacji w polskojęzycznej strefie platformy Twitter na podstawie wybranych kont. Dokonano analizy porównawczej charakterystyk tej klasy kont, przeciwstawiając je do kont należących do profesjonalnych serwisów informacyjnych. Z analizy wynikało, że najbardziej obiecującą charakterystyką możliwą do klasyfikacji publikowanych przez badane konta informacji są dane o użytkownikach udostępniających posty. Na tej podstawie dokonano badania automatycznej klasyfikacji postów z wykorzystaniem dwóch różnych metod. Dla wystarczająco dużego zbioru uczącego otrzymano precyzję ponad 90\% dla obu algorytmów. Wynik ten prawdopodobnie mógłby być jeszcze lepszy, gdyby dysponowano informacją o większej ilości użytkowników. Nie było to jednak możliwe ze względu na ograniczenia techniczne źródła tych danych. 
\par
Widać więc, że udało się uzyskać bardzo obiecujące wyniki. Przeprowadzona klasyfikacja udowadnia, że możliwe jest etykietowanie treści na podstawie ich kontekstu jakim są udostępniający je użytkownicy. Przedstawiona praca skupia się na części problemu jakim są nierzetelne informacje w polskiej przestrzeni serwisów społecznościowych i pokazuje, że w tej dziedzinie istnieje obiecująca ścieżka rozwoju dla technik automatycznej klasyfikacji. Dodatkowo omówiono istniejące inicjatywy i przedstawiono własne badania które mogą być podstawą do dalszego rozwoju prac podejmujących temat nierzetelnych informacji. 
