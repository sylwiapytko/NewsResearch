\newpage % Rozdziały zaczynamy od nowej strony.
\section{Przegląd istniejących rozwiązań klasyfikacji informacji}
\subsection{Rozwiązania manualne}
\subsubsection{Praca ekspertów}
Badanie rzetelności faktów zawartych w artykułach lub w wypowiedziach osób publicznych jest często zadaniem skomplikowanym oraz czasochłonnym. Są jednak inicjatywy poświęcone wykrywaniu fałszu lub niepełnej prawdziwości w publikowanych informacjach. Jednymi z najbardziej popularnych stron zajmujących się taką weryfikacją są Politifact\footnote{\url{https://www.politifact.com}} oraz Snopes\footnote{\url{https://www.snopes.com}} . Podejmują one głównie tematy związane ze polityką Stanów Zjednoczonych.  Ich działanie polega na zatrudnianiu niezależnych dziennikarzy, których zadaniem jest wyszukiwanie źródeł weryfikujących wiarygodność informacji podawanych w sieci. Oceniają oni daną informację według ustalonej klasyfikacji. Aby być samemu być wiarygodnym zawsze podają źródła zarówno sprawdzanej informacji jak i materiałów użytych do wydania oceny. Dzięki temu użytkownik sam może zweryfikować swoją opinię. 
\par To rozwiązanie mimo oczywistych plusów, jakimi jest rzetelność i przejrzystość weryfikacji wierzytelności ma też znaczące minusy. Wymaga ono dużych nakładów pracy wykonanej przez specjalnie zatrudnionych w tym celu ekspertów. Metoda ta jest również czasochłonna zwłaszcza przy niejasnych przypadkach. Zwiększa to ryzyko, że kłamliwa informacja zostanie rozpowszechniona wśród większej liczby osób zanim zostanie zweryfikowana. Ponadto działanie takiego rozwiązania opiera się na założeniu, że użytkownicy czytają jedną z weryfikujących stron, co wymaga większego zaangażowania i może powodować pewną niedogodność dla standardowego odbiorcy. 
\subsubsection{Crowdsourcing}
Inną stosowaną metodą na wykrywanie fałszu jest crowdsourcing. Istnieją rozwiązania, które zbierają opinie użytkowników na temat prawdziwości lub nieszczerości konkretnych wypowiedzi umieszczanych w Internecie. Dzięki temu mogą one ostrzegać innych użytkowników przed zbytnią ufnością do czytanego źródła. Takie podejście wykorzystuje na przykład projekt fakenewsdetector.org\footnote{\url{https://fakenewsdetector.org/en}}. Jest to projekt opensource, który stworzył wtyczkę internetową o tej samej nazwie. Połączył on opisaną wyżej metodę jako zbieranie danych do uczenia maszynowego. Dzięki temu uczy się klasyfikować informacje świeżo opublikowane, które nie zdążyły być jeszcze przeczytane przez żadnego użytkownika. 
\par Fiskkit\footnote{\url{https://fiskkit.com}} jest platformą, która daje większe możliwości dyskusji nad artykułem niż przeciętne medium z artykułami. Platforma nie zawiera własnych artykułów, ale pozwala użytkownikom na importowanie ich z dowolnego źródła i daje możliwość dyskusji bezpośrednio nad każdym osobnym zdaniem w danym tekście. Użytkowni-cy mogą więc wskazywać dokładne fragmenty które uważają za fałszywe lub zbyt uproszczone.
\par Dużym minusem wykorzystania corwdsourcingu w ocenianiu prawdziwości artykułów powinien być brak zaufania do mas użytkowników. Osoby udzielające swojej opinii mogą być skrajnie stronnicze tak samo jak są przy udostępnianiu niepewnych informacji na portalach społecznościowych.

\subsection{Rozwiązania automatyczne}
Poniżej przedstawię kilka wybranych rozwiązań informatycznych działających w kierunku wykrywania fałszu lub stronniczości w artykułach i mediach społeczno-ściowych. 
Warto na początku zaznaczyć, że większość istniejących prac skupia się na tre-ściach intencjonalnie fałszywych, odrzucając teorie spiskowe, plotki, ponieważ te z definicji są trudniejsze do określenia czy są całkowicie fałszywe czy zawierają praw-dę. Z rozpoznawania powinno się też wyłączyć satyrę, ponieważ ta działa na innych prawach. Portale publikujące teksty satyryczne zazwyczaj bezpośrednio podają do wiadomości użytkownika, że zawierają treści humorystyczne i informacje w nich zawarte nie powinny być traktowane poważnie.

\subsubsection{Style-based}
Rozważania nad używaniem metod przetwarzania języka naturalnego wywodzi się z założenia, że artykuły, które szerzą intencjonalną dezinformację różnią się stylem od artykułów prawdziwych. Może się to ujawniać na przykład w bardziej emocjonalnym słownictwie, poruszanych tematach lub skrajnej stronniczości.
\par Częstym zabiegiem wykorzystywanym do szerzenia dezinformacji w artykułach jest wykorzystywanie chwytliwego tytułu. Taka metoda często ma za zadanie zwabić czytelnika na kliknięcie w artykuł (tzw. clickbait). Oprócz takiego zastosowania równie często zdarza się, że sam tytuł może wyrażać fałszywe stwierdzenie natomiast treść artykułu naprostowuje je w stronę prawdy, użytkownik jednak nie dowie się o omylności tytułu, jeśli nie zdecyduje się przeczytać całości.  W 2017 zorganizowany został konkurs FakeNewsChallange\footnote{\url{http://www.fakenewschallenge.org}},  którego pierwszym etapem było stworzenie rozwiązania wykrywającego w jaki sposób treść artykułu odnosi się do jego tytułu. Może on bowiem zgadzać się lub nie zgadzać z tytułem. Artykuł może też omawiać dany temat ale nie wyrażać własnej opinii, możliwe jest też że treść dotyczy innego tematu niż tytuł. Rozwiązanie, które zostało ocenione najlepiej zostało stworzone przez grupę o nazwie Solat in the Swen. Stworzyli oni model oparty na połączeniu drzew decyzyjnych oraz głębokich sieci neuronowych.  

\subsubsection{Style-based}