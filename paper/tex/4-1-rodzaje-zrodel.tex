
\newpage
\section{Badanie wiarygodności informacji}
W tym rozdziale zostaną przedstawione i opisane badania wiarygodności informacji skupiające się na informacjach związanych z Polską i Unią Europejską. Jako wiarygodność informacji rozumie się prawdopodobieństwo, że dana informacja jest prawdziwa. Wiarygodność źródła oznacz prawdopodobieństwo, że informacje które ono udostępnia są prawdziwe. Jako synonim do małego prawdopodobieństwa prawdziwości będą używane słowa nierzetelny, mało wiarygodny, nieprawdziwy lub dezinformujący. 
\subsection{Rodzaje źródeł informacji}
Centrum badania opinii społecznej CBOS przeprowadziło w\,kwietniu 2019 roku badania dotyczące wiarygodności mediów w\,Polsce. Zostały one przeprowadzone metodą wywiadów bezpośrednich na reprezentatywnej próbie losowej dorosłych liczącej 1064 osoby. CBOS przeprowadziło takie badania już po raz drugi, dzięki czemu można porównać jakie zmiany w\,opinii publicznej nastąpiły w\,ciągu dwóch ostatnich lat\cite{CBOSWiarygodnoscMediow2019}.
\par 
Jednym z\,punktów zainteresowań badania była informacja co jest głównym źródłem czerpania informacji o wydarzeniach w\,kraju i\,na świecie Polaków.  Z\,uzyskanych odpowiedzi wynika, że wzrosło znaczenie Internetu używanego w\,takim celu z\,21\% w\,2017 roku do 27\% w\,2019 roku. Jednocześnie zmalała o sześć procent liczba osób deklarująca telewizję jako główne źródło informacji. Inne źródła takie jak radio i\,prasa są najważniejszym źródłem informacji dla relatywnie niewielkiej grupy. Dodatkowo patrząc na zróżnicowanie społeczne odbiorców poszczególnych mediów można zauważyć, że bardzo duże znaczenie ma wiek. W\,grupie młodych dorosłych (osoby w\,wieku 18-24 lata) 60\% deklaruje, że ich głównym źródłem pobierania informacji jest Internet. Patrząc na takie statystyki śmiało można założyć, że znaczenie Internetu w\,dostarczaniu informacji będzie nadal rosło.