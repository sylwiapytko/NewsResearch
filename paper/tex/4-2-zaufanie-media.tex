
\subsubsection{Zaufanie do informacji w\,mediach}
Na zlecenie Komisji Europejskiej międzynarodowy projekt badania opinii publicznej Eurobarometer w\,2018 roku przeprowadził badania na temat fałszywych informacji w\,Internecie\cite{Eurobarometer4642018}. W\,badaniu pytano obywateli wszystkich krajów Unii Europejskiej w\,tym również Polski jakim stopniu ufają różnym typom mediów. Jako zaufanie uznaje się odpowiedź „całkowicie ufam” oraz „raczej ufam” po-daną przez respondenta na pytanie. Polacy mają największe zaufanie do radia 63\% a\,następnie do drukowanej prasy oraz telewizji (odpowiednio 55\% i\,54\%). Patrząc na media publikujące w\,Internecie największym zaufaniem cieszą się portale informacyjne 45\% a\,najmniej respondentów, bo 34\% ma zaufanie do informacji pobieranych poprzez media społecznościowe oraz aplikacje służące do komunikacji. 
\par
Jednak jak wynika z\,poprzednich badań przeprowadzonych przez Eurobarometr prawie jedna piąta użytkowników Internetu w\,Polsce (19\%) pobiera informacje o wiadomościach korzystając z\,mediów społecznościowych. Warto przy tym zwrócić uwagę, że tylko połowa osób kliknie w\,link, aby przeczytać całość artykułu na jego oryginalnej stronie internetowej\cite{Eurobarometer2016}. Takie zachowanie użytkowników umożliwia nierzetelnym stronom na manipulacje informacjami. Zdarza się bowiem, że tytuł przedstawia informację w\,fałszywy sposób a\,dopiero treść artykułu wyjaśnia jak sytuacja naprawdę wygląda. 
\subsection{Fałszywe informacje w\,Internecie}
Badania skupiające się na opinii Polskich internautów o dezinformacji w\,sieci przeprowadził państwowy instytut badawczy NASK na przełomie marca i\,kwietnia 2019\cite{NASKBezpieczneWybory2019}. Zapytano grupy 1000 internautów czy w\,przeciągu 6 ostatnich miesięcy spotkali się w\,Internecie z\,informacjami, które według ich opinii mogły być sfałszowane lub zmanipulowane. Z\,tych badań wynika, że ponad połowa ankietowanych (56\%) spotkała się z\,takim zjawiskiem w\,Internecie w\,przeciągu ostatniego pół roku od chwili zadanego pytania, w\,tym prawie co czwarta osoba zauważa jakąś manipulację informacją codziennie lub kilka razy w\,tygodniu. Tylko 11\% ankietowanych twierdzi, że nie spotkało się z\,zmanipulowanymi informacjami. Pozostałe osoby nie potrafiły odpowiedzieć na to pytanie w\,twierdzący lub zaprzeczający sposób. Część wyników przedstawiono za pomocą diagramu kołowego na rysunku \ref{fig:NASKwyniki}.
\begin{figure}[!h]
	
	\centering \includegraphics[width=0.9\linewidth]{img/NASKwyniki.PNG}
	\caption{Wybrane wyniki z\,badania opinii Polaków na temat dezinformacji w\,internecie. Opracowanie własne na podstawie źródła: \cite{NASKBezpieczneWybory2019}}
	\label{fig:NASKwyniki}
\end{figure}
\par
\par
Najwięcej respondentów (30\%) jako najczęściej spotykaną przez nich formę dezinformacji i\,manipulacji w\,Internecie określiło fake news, czyli szeroko pojęte fałszywe informacje.  Następnie wskazywano zmanipulowane zdjęcia oraz tzw. trolling czyli celowe prowokowanie do kłótni na forach internetowych poprzez publikowanie nieprawdziwych albo emocjonalnych treści. Jedna piąta ankietowanych nie była jednak w\,stanie powiedzieć jaki typ dezinformacji lub manipulacji najczęściej spotykają.
\par
Tak częste występowanie fałszywych informacji w\,sieci wiąże się z\,tym, że część użytkowników Internetu intencjonalnie lub nieintencjonalnie wchodzi z\,nimi w\,aktywną interakcję przesyłając dalej takie treści lub wyrażając aprobatę poprzez klikanie ‘lubię to’ przy takich publikacjach. Do takiego zachowania w\,czasie ostatniego pół roku przyznało się łącznie prawie 20\% ankietowanych w\,tym 12\% deklaruje, że zrobiło to nie mając świadomości o fałszywości zawartych tam informacji\cite{NASKBezpieczneWybory2019}. 

