\subsection{Weryfikacja informacji} \label{badanie-wiarygodnosci-w-internecie}
Rozprzestrzenianie  fałszywych informacji w\,Internecie potęguje fakt, że większość osób nigdy lub tylko sporadycznie weryfikuje czy informacje, które czytają są wiarygodne. Tylko 25\% badanych zadeklarowało, że kilkakrotnie w\,przeciągu ostatniego pół roku sprawdziło wiarygodność czytanej informacji w\,sieci pod kątem treści, źródła lub wiarygodności informacji zawartych na profilu w\,mediach społecznościowych\cite{NASKBezpieczneWybory2019}. Osoby, które najczęściej weryfikują czytane treści internetowe to takie, które często napotykają zmanipulowane wiadomości o tematyce ideologiczno-kulturowej.
\par
Sposoby jakie respondenci najczęściej używają w\,celu oceny wiarygodności informacji znalezionych w\,Internecie można podzielić na 3 grupy metod. Są nimi:
\begin{itemize}
    \item Rzetelność dziennikarska – odbiorcy patrzą na to czy do publikowanej treści podawane są źródła, na których ona się opiera oraz czy zawiera różne punkty widzenia. 
    \item Osobiste zaufanie – bardziej ufamy informacjom, które pochodzą ze źródła, do którego mamy już zbudowane zaufanie lub autorem jest osoba której ufamy. Do tego dołącza również przeświadczenie, że jeśli daną informacje publikuje albo poleca osoba, którą znamy będziemy bardziej skłonny zaufać i\,uwierzyć w\,czytane treści.
    \item Społeczny dowód słuszności – polega ona na tym, że dużą wiarygodność otrzymują informacje, które pochodzą z\,dużego i\,popularnego medium albo gdy dana informacja przekazywana jest przez wiele osób lub mediów jednocześnie.   
\end{itemize}

