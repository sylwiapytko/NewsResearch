
\subsection{Klasyfikacja wiarygodności źródeł}
Aby powstał jakikolwiek automatyczny sposób wykrywania fałszywych informacji w\,Internecie na podstawie jego kontekstu należy najpierw przygotować odpowiednią klasyfikację. Ocenianie każdego opublikowanego artykułu lub postu z\,wybranego zbioru jest zadaniem niezwykle żmudnym. Należy więc utworzyć mechanizm grupowania ocenianych treści. W\,jednym ze sposobów można uznać, że jeśli konkretne źródło tworzące lub udostępniające treści uznamy za niewiarygodne, można założyć, że wszystkie treści publikowane przez to źródło są niewiarygodne. Takie postępowanie przyjęli twórcy artykułu „Some like it Hoax”\cite{tacchini2017some}, badający zachowania użytkowników portalu Facebook względem postów wystawianych przez strony naukowe oraz o tematyce konspiracyjnej. Podobne podziały dla mediów amerykańskich można znaleźć na wyspecjalizowanych stronach takich jak politifact.com\footnote{\url{https://www.politifact.com/punditfact/article/2017/apr/20/politifacts-guide-fake-news-websites-and-what-they/}},  mediabiasfactcheck.com\footnote{\url{https://mediabiasfactcheck.com}}  lub innych zbiorach stworzonych specjalnie do tego celu. Są to strony wyspecjalizowane w\,sprawdzaniu prawdziwości treści i\,wykrywaniu dezinformacji. 
\par
Większość działań podjętych w\,celu wykrycia nierzetelnych stron internetowych publikujących nieprawdziwe informacje skupia się na treściach powiązanymi ze Stanami Zjednoczonymi Ameryki. Niewiele podobnych inicjatyw istnieje w\,Europie. Nie dziwi to jednak ponieważ, ocenianie całych portali internetowych pod względem ich wiarygodności jest niezwykle wrażliwym tematem i\,jest obciążone olbrzymią odpowiedzialnością. 
\subsubsection{Nierzetelne źródła medialne w\,Europie} \label{nierzetelne-zrodla-eu}
Na skalę europejską takiego działania podjął się zespół Instytutu Informatyki w\,Oxfo- rdzie jako część projektu o nazwie Computational Propaganda project COMPROP. Zespół zajmuje się między innymi analizą w\,jaki sposób automatyzacja działań w mediach społecznościowych wpływa na rozprzestrzenianie treści związanych z\,polityką, mową nienawiści i\,dezinformacji oraz jak to może wpłynąć na manipulowanie opinią publiczną. 
\par
Jeden z\,projektów tego zespołu „Junk news aggregator” powstał z\,intencją pomocy naukowcom i\,dziennikarzom w\,kolejnych badaniach nad problemem dezinformacji w\,Inter- necie. Projekt ten zbierał posty publikowane na platformie Facebook przez strony uznane przez zespół jako publikujące śmieciowe informacje.  W\,swojej pierwszej pracy na temat tego projektu\cite{liotsiou2019junk} wyodrębniono strony uznane przez nich jako śmieciowe których publikowane treści dotyczyły wyborów w\,Stanach Zjednoczonych przeprowadzonych listopadzie 2018. Następnie, aby dać możliwość obserwowania zmanipulowanych informacji dotyczących wyborów do Parlamentu Europejskiego w\,maju 2019 dołączono również kilkadziesiąt portali z\,kilku krajów Europy w\,tym również z\,Polski. Dzięki temu, dostępna jest lista 13-tu\,polskich portali informacyjnych które zespół COMPROP uznał za nierzetelne. 
\subsubsection{Metoda oceny źródeł} \label{metoda-oceny-zrodel}
Przypisywanie kategorii stronom internetowym wykonane było przez ekspertów przygotowanych dla każdego kraju i\,było oparte na stworzonej wcześniej typologii\cite{neudertpolarization2018}. W\,przy- padku nierzetelnych źródeł medialnych jest pięć kryteriów, jeśli źródło spełnia przynajmniej trzy z\,nich było zakwalifikowywane jako należące do kategorii ‘śmieciowe’. Poniżej przedstawiam opis tych pięciu kategorii oraz dodatkowej szóstej będącej zależnej od poprzednich:
\begin{itemize}
    \item Profesjonalizm – są to źródła, które nie spełniają standardów profesjonalnego dziennikarstwa. Często nie podają informacji na temat autorów, redaktorów czy wydawców. Brakuje im przejrzystości i\,odpowiedzialności za publikowane treści. Nie publikują korekt do obalonych informacji.
    \item Styl – portale które używają emocjonalnego języka, hiperbol lub mylących tytułów. Wstawiają dużo niepowiązanych zdjęć, często wzbudzających emocje w\,odbiorcach.
    \item Wiarygodność – publikujące fałszywe informacje lub teorie spiskowe. Nie sprawdzają innych źródeł ani prawdziwości udostępnianych przez siebie informacji.
    \item Stronniczość – publikowane przez nich treści są silnie stronnicze ideologicznie lub politycznie. Raportowane informacje zazwyczaj zawierają opiniotwórczy komentarz.
    \item Imitacja – są to portale które w\,swojej formie przypominają rzetelne źródło informacyjne upodabniając używaną czcionkę i\,styl. Powołują się na źródła uznawane za wiarygodne, ale publikują tylko własną opinię lub bezwartościowe treści.
    \item Agregacja – strony, które udostępniają artykuły stworzone przez źródła, które zostały na podstawie poprzednich kategorii uznane za śmieciowe.
\end{itemize}