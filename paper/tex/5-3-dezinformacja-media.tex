
\subsection{Informacja i\,dezinformacja w\,mediach społecznościowych}
\label{istniejace-badania-twitter}
Najbardziej popularnym i\,najczęściej używanym portalem społecznościowym w\,Polsce jest Facebook. Jest on drugą najczęściej odwiedzaną domeną zaraz po google.com. Prawie trzy czwarte polskich internautów, czyli 21 milionów Polaków jest jego użytkownikami a\,codziennie średnio ponad 7 milionów Polaków odwiedza ten portal\cite{GemiusInternet2019}. Drugim najbardziej popularnym portalem społecznościowym jest Instagram będący serwisem do publikowania zdjęć a\,na trzecim miejscu plasuje się Twitter z\,liczbą prawie 5 milionów polskich użytkowników\cite{GemiusSerwisy2019}. Mimo, że Twitter jest rzadziej używany przez ogół internautów jest ważną platformą dla polityków, działaczy społecznych oraz dziennikarzy\cite{gorwa2017computational}. Dzięki swojej formie mikroblogów jest wygodnym narzędziem do przekazania publiczności informacji, opinii lub komentarza w\,krótkiej formie jaką jest Tweet.
\subsubsection{Badanie typów informacji na Twitterze}
Przed wyborami do Parlamentu Europejskiego COMPROP przeprowadziło badania na temat typów informacji rozprzestrzenianych na platformach społecznościowych takich jak Twitter i\,Facebook\cite{marchal2019junk}. Pod uwagę wzięto siedem grup językowych w\,tym również polski. Dzięki odpowiednim hashtagom zebrano Tweety związane z\,wyborami a\,następnie wyodrębniono linki URL. Tak uzyskane źródła zakwalifikowano do pięciu typów: Profesjonalne portale informacyjne, profesjonalne źródła rządowe, śmieciowe portale, inne portale informacyjne oraz inne źródła. Dzięki temu można było zbadać które źródła mają największą popularność wśród użytkowników mediów społecznościowych. Okazało się, że w\,zebranym zbiorze Tweetów tylko 4\% zawiera w\,sobie link do źródła oznaczonego jako śmieciowe. Jednak w\,podzbiorze polskojęzycznym było to aż 20\% badanych Tweetów. Należy pamiętać, że nie brano pod uwagę kontekstu z\,jakim opublikowany był każdy Tweet, możliwym jest, że mógł zawierać sprostowanie lub niezgodę z\,zawartą w\,podlinkowanym artykule. Nie mniej oznacza to, że według powyższych badań co piąty polskojęzyczny Tweet zawierał link do nierzetelnego źródła.  
\subsubsection{Rozprzestrzenianie dezinformacji na Twitterze}
Rozprzestrzenianie się nieprawdziwych informacji i\,plotek na platformie Twitter jest poważnym problemem. Porównując popularność publikowanych plotek zauważono, że te które okazują się zawierać nieprawdę mają o 70\% większe prawdopodobieństwo ponownego udostępnienia przez innych użytkowników w\,porównaniu do tych które mówią prawdę\cite{vosoughi2018spread}. Tweety zawierające nieprawdę rozprzestrzeniają się szybciej i\,docierają do większej ilości użytkowników. Zauważono, że użytkownicy częściej udostępniają wiadomości, które są nowe i\,oryginalne, a\,fałszywe plotki często takie właśnie są. 
\par
Podczas innego badania przyjrzano się problemowi botów, czyli kont sterowanych nie przez człowieka, ale przez algorytm\cite{gorwa2017computational}. Przy użyciu metody heurystycznej ze zbioru ponad 10 000 kont wyodrębniono 500 kont przy których istnieje podejrzenie, że nie są autentyczne. Okazuje się, że te 5\% kont opublikowało 33\% zebranych Tweetów o\,tematyce politycznej. Pokazuje to jak duży wpływ na treści publikowane na platformach społecznościowych mogą mieć konta stworzone specjalnie w\,celu rozpowszechniania konkretnych treści. 
