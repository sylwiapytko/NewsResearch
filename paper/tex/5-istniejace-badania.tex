\newpage
\section{Badanie wiarygodności informacji w\,Internecie}
\subsection{Weryfikacja informacji}
Rozprzestrzenianie się fałszywych informacji w\,Internecie potęguje fakt, że większość osób nigdy lub tylko sporadycznie weryfikuje czy informacje, które czytają są wiarygodne. Tylko 25\% badanych zadeklarowało, że kilkakrotnie w\,przeciągu ostatniego pół roku sprawdziło wiarygodność czytanej informacji w\,sieci pod kątem treści albo źródła lub wiarygodności informacji zawartych na profilu w\,mediach społecznościowych\cite{NASKBezpieczneWybory2019}. Osoby, które najczęściej weryfikują czytane treści internetowe to takie które często napotykają zmanipulowane wiadomości o tematyce ideologiczno-kulturowej.
\par
Sposoby jakie respondenci najczęściej używają w\,celu oceny wiarygodności informacji znalezionych w\,Internecie można podzielić na 3 grupy metod. Są nimi:
\begin{itemize}
    \item Rzetelność dziennikarska – odbiorcy patrzą na to czy do publikowanej treści podawane są źródła, na których ona się opiera oraz czy zawiera różne punkty widzenia. 
    \item Osobiste zaufanie – bardziej ufamy informacjom, które pochodzą ze źródła, do którego mamy już zbudowane zaufanie lub autorem jest osoba której ufamy. Do tego dołącza również przeświadczenie, że jeśli daną informacje publikuje albo poleca osoba, którą znamy będziemy bardziej skłonny zaufać i\,uwierzyć w\,czytane treści.
    \item Społeczny dowód słuszności – polega ona na tym, że dużą wiarygodność otrzymują informacje, które pochodzą z\,dużego i\,popularnego medium albo gdy dana informacja przekazywana jest przez wiele osób lub mediów jednocześnie.   
\end{itemize}

\subsection{Klasyfikacja wiarygodności źródeł}
Aby powstał jakikolwiek automatyczny sposób wykrywania fałszywych informacji w\,Internecie na podstawie jego kontekstu należy najpierw przygotować odpowiednią klasyfikację. Ocenianie każdego opublikowanego artykułu lub postu z\,wybranego zbioru jest zadaniem niezwykle żmudnym. Należy więc utworzyć mechanizm grupowania ocenianych treści. W\,jednym ze sposobów można uznać, że jeśli konkretne źródło tworzące lub udostępniające treści uznamy za niewiarygodne, można założyć, że wszystkie treści publikowane przez to źródło są niewiarygodne. Takie postępowanie przyjęli twórcy artykułu „Some like it Hoax”\cite{tacchini2017some}, badający zachowania użytkowników portalu Facebook względem postów wystawianych przez strony naukowe oraz o tematyce konspiracyjnej. Podobne podziały dla mediów amerykańskich można znaleźć na wyspecjalizowanych stronach takich jak politifact.com\footnote{\url{https://www.politifact.com/punditfact/article/2017/apr/20/politifacts-guide-fake-news-websites-and-what-they/}},  mediabiasfactcheck.com\footnote{\url{https://mediabiasfactcheck.com}}  lub innych zbiorach stworzonych specjalnie do tego celu. Są to strony wyspecjalizowane w\,sprawdzaniu prawdziwości treści i\,wykrywaniu dezinformacji. 
\par
Większość działań podjętych w\,celu wykrycia nierzetelnych stron internetowych publikujących nieprawdziwe informacje skupia się na treściach powiązanymi ze Stanami Zjednoczonymi Ameryki. Niewiele podobnych inicjatyw istnieje w\,Europie. Nie dziwi to jednak ponieważ, ocenianie całych portali internetowych pod względem ich wiarygodności jest niezwykle wrażliwym tematem i\,jest obciążone olbrzymią odpowiedzialnością. 
\subsubsection{Nierzetelne źródła medialne w\,Europie}
Na skalę europejską takiego działania podjął się zespół Instytutu Informatyki w\,Oxfo- rdzie jako część projektu o nazwie Computational Propaganda project COMPROP. Zespół zajmuje się między innymi analizą w\,jaki sposób automatyzacja działań na mediach społecznościowych wpływa na rozprzestrzenianie treści związanych z\,polityką, mową nienawiści i\,dezinformacji oraz jak to może wpłynąć na manipulowanie opinią publiczną. 
\par
Jeden z\,projektów tego zespołu „Junk news aggregator” powstał z\,intencją pomocy naukowcom i\,dziennikarzom w\,kolejnych badaniach nad problemem dezinformacji w\,Inter- necie. Projekt ten zbierał posty publikowane na platformie Facebook przez strony uznane przez zespół jako publikujące śmieciowe informacje.  W\,swojej pierwszej pracy na temat tego projektu\cite{liotsiou2019junk} wyodrębniono strony uznane przez nich jako śmieciowe których publikowane treści dotyczyły wyborów w\,Stanach Zjednoczonych przeprowadzonych listopadzie 2018. Następnie, aby dać możliwość obserwowania zmanipulowanych informacji dotyczących wyborów do Parlamentu Europejskiego w\,maju 2019 dołączono również kilkadziesiąt portali z\,kilku krajów Europy w\,tym również z\,Polski. Dzięki temu, dostępna jest lista 13-tu\,polskich portali informacyjnych które zespół COMPROP uznał za nierzetelne. 
\subsubsection{Metoda oceny źródeł}
Przypisywanie kategorii stronom internetowym wykonane było przez ekspertów przygotowanych dla każdego kraju i\,było oparte na stworzonej wcześniej typologii\cite{neudertpolarization2018}. W\,przy- padku nierzetelnych źródeł medialnych jest pięć kryteriów, jeśli źródło spełnia przynajmniej trzy z\,nich było zakwalifikowywane jako należące do kategorii ‘śmieciowe’. Poniżej przedstawiam opis tych pięciu kategorii oraz dodatkowej szóstej będącej zależnej od poprzednich:
\begin{itemize}
    \item Profesjonalizm – są to źródła, które nie spełniają standardów profesjonalnego dziennikarstwa. Często nie podają informacji na temat autorów, redaktorów czy wydawców. Brakuje im przejrzystości i\,odpowiedzialności za publikowane treści. Nie publikują korekt do obalonych informacji.
    \item Styl – portale które używają emocjonalnego języka, hiperbol lub mylących tytułów. Wstawiają dużo niepowiązanych zdjęć, często wzbudzających emocje w\,odbiorcach.
    \item Wiarygodność – publikujące fałszywe informacje lub teorie spiskowe. Nie sprawdzają innych źródeł ani prawdziwości udostępnianych przez siebie informacji.
    \item Stronniczość – publikowane przez nich treści są silnie stronnicze ideologicznie lub politycznie. Raportowane informacje zazwyczaj zawierają opiniotwórczy komentarz.
    \item Imitacja – są to portale które w\,swojej formie przypominają rzetelne źródło informacyjne upodabniając używaną czcionkę i\,styl. Powołują się na źródła uznawane za wiarygodne, ale publikują tylko własną opinię lub bezwartościowe treści.
    \item Agregacja – strony, które udostępniają artykuły stworzone przez źródła, które zostały na podstawie poprzednich kategorii uznane za śmieciowe.
\end{itemize}

\subsection{Informacja i\,dezinformacja w\,mediach społecznościowych}
Najbardziej popularnym i\,najczęściej używanym portalem społecznościowym w\,Polsce jest Facebook. Jest on drugą najczęściej odwiedzaną domeną zaraz po google.com. Prawie trzy czwarte polskich internautów, czyli 21 milionów Polaków jest jego użytkownikami a\,codziennie średnio ponad 7 milionów Polaków odwiedza ten portal\cite{GemiusInternet2019}. Drugim najbardziej popularnym portalem społecznościowym jest Instagram będący serwisem do publikowania zdjęć a\,na trzecim miejscu plasuje się Twitter z\,liczbą prawie 5 milionów polskich użytkowników\cite{GemiusSerwisy2019}. Mimo, że Twitter jest rzadziej używany przez ogół internautów jest ważną platformą dla polityków, działaczy społecznych oraz dziennikarzy\cite{gorwa2017computational}. Dzięki swojej formie mikroblogów jest wygodnym narzędziem do przekazania publiczności informacji, opinii lub komentarza w\,krótkiej formie jaką jest Tweet.
\subsubsection{Badanie typów informacji na Twitterze}
Przed wyborami do Parlamentu Europejskiego COMPROP przeprowadziło badania na temat typów informacji rozprzestrzenianych na platformach społecznościowych takich jak Twitter i\,Facebook\cite{marchal2019junk}. Pod uwagę wzięto siedem grup językowych w\,tym również polski. Dzięki odpowiednim hashtagom zebrano Tweety związane z\,wyborami a\,następnie wyodrębniono linki URL. Tak uzyskane źródła zakwalifikowano do pięciu typów: Profesjonalne portale informacyjne, profesjonalne źródła rządowe, śmieciowe portale, inne portale informacyjne oraz inne źródła. Dzięki temu można było zbadać które źródła mają największą popularność wśród użytkowników mediów społecznościowych. Okazało się, że w\,zebranym zbiorze Tweetów tylko 4\% zawiera w\,sobie link do źródła oznaczonego jako śmieciowe. Jednak w\,podzbiorze polskojęzycznym było to aż 20\% badanych Tweetów. Należy pamiętać, że nie brano pod uwagę kontekstu z\,jakim opublikowany był każdy Tweet, możliwym jest, że mógł zawierać sprostowanie lub niezgodę z\,zawartą w\,podlinkowanym artykule. Nie mniej oznacza to, że według powyższych badań co piąty polskojęzyczny Tweet zawierał link do nierzetelnego źródła.  
\subsubsection{Rozprzestrzenianie dezinformacji na Twitterze}
Rozprzestrzenianie się nieprawdziwych informacji i\,plotek na platformie Twitter jest poważnym problemem. Porównując popularność publikowanych plotek zauważono, że te które okazują się zawierać nieprawdę mają o 70\% większe prawdopodobieństwo ponownego udostępnienia przez innych użytkowników w\,porównaniu do tych które mówią prawdę\cite{vosoughi2018spread}. Tweety zawierające nieprawdę rozprzestrzeniają się szybciej i\,docierają do większej ilości użytkowników. Zauważono, że użytkownicy częściej udostępniają wiadomości, które są nowe i\,oryginalne, a\,fałszywe plotki często takie właśnie są. 
\par
Podczas innego badania przyjrzano się problemowi botów, czyli kont sterowanych nie przez człowieka, ale przez algorytm\cite{gorwa2017computational}. Przy użyciu metody heurystycznej ze zbioru ponad 10 000 kont wyodrębniono 500 kont przy których istnieje podejrzenie, że nie są autentyczne. Okazuje się, że te 5\% kont opublikowało 33\% zebranych Tweetów o\,tematyce politycznej. Pokazuje to jak duży wpływ na treści publikowane na platformach społecznościowych mogą mieć konta stworzone specjalnie w\,celu rozpowszechniania konkretnych treści. 
