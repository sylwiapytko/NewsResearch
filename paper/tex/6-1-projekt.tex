\newpage
\section{Projekt badania}
\subsection{Opis}
W ramach tej pracy stworzono projekt mający na celu zbadać skalę problemu jaką są nierzetelne informacje w\,mediach społecznościowych oraz sprawdzenie skuteczności automatycznej klasyfikacji informacji w\,oparciu o dane kontekstowe.  W\,tym celu skupiono się na polskojęzycznych mediach publikujących na platformie Twitter oraz grupie ich odbiorców. Do analizy zostały wybrane media mogące być zaklasyfikowanymi ze względu na ich rzetelność. Ustalenie klas danych testowych oparto na pracy zespołu uniwersytetu Oxford wskazującej nierzetelne media polskojęzyczne. Przeciwstawiono je mediom profesjonalnym oraz o dużej liczbie odbiorców tzw. Mainstream media.
\subsubsection{Cel projektu}
Pierwszym celem tego projektu jest zbadanie jak popularne są treści publikowane przez nierzetelne media w\,porównaniu do tych mainstreamowych. Zostanie to dokonane poprzez badanie częstotliwości zamieszczania nowych treści, wielkość grupy odbiorców i\,przede wszystkim skalę rozprzestrzeniania się tych informacji biorąc pod uwagę bezpośrednie udostępnienia użytkowników. Zostaną również przeanalizowane wzajemne powiązania między badanymi kontami oraz domeny, które są zamieszczane w\,ich postach.
\par
Drugim celem jest sprawdzenie możliwości automatycznej klasyfikacji postowanych informacji na podstawie danych o użytkownikach je udostępniających. Badanie to jest oparte na pracy Some like it hoax, w\,której badając posty na facebooku i\,użytkowników wchodzących z\,nimi w\,interakcje otrzymało bardzo wysokie wyniki klasyfikacji. W\,celu odtworzenia badania na danych polskojęzycznych zostaną użyte te same algorytmy. Badanie ma wykazać czy dla omawianych danych możliwa jest taka klasyfikacja. Dodatkowo, zostanie zbadane jaka liczba danych wystarczy, aby taką klasyfikację przeprowadzić z\,zadowalającą dokładnością. 
