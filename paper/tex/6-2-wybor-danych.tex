
\subsection{Wybór danych do analizy} \label{wybor-danych}
Istniejące prace podejmujące temat analizy i\,próby klasyfikacji nierzetelnych informacji opiera swoje badania na podziale binarnym całych źródeł medialnych\cite{marchal2019junk}\cite{tacchini2017some}. Jak opisano w\,powyższych rozdziałach \ref{rozwiazaniamanualne} manualna klasyfikacja prawdziwości informacji jest zadaniem długotrwałym i\,wymagającym pracy ekspertów. Z\,tego powodu podejmuje się uogólnienie klasyfikujące całe źródła. To uogólnienie opiera się na założeniu, że informacje publikowane przez źródło zakwalifikowane jako nierzetelne są nierzetelne, i\,tak samo z\,źródłami wiarygodnymi. 
\par
Wzorując się na innych pracach zdecydowano przyjrzeć się źródłom medialnym mogących być zaklasyfikowanymi do konkretnej kategorii. Okazało się być to jednym z\,najtrudniejszych zadań w\,całym procesie podjętym w\,tej pracy. Według najlepszej wiedzy autora, w\,momencie tworzenia tej pracy nie istnieją żadne naukowe inicjatywy podejmujące klasyfikację polskich mediów, oprócz prac prowadzonych przez zespół COMPROP na uniwersytecie w\,Oxfordzie. Prace te zostały dokładniej opisane w\,rozdziale \ref{badaniewiarygodnosciwinternecie}. Niestety liczba opublikowanych Polskich źródeł w\,klasyfikacji jest niewystarczająca do przeprowadzenia analizy porównawczej oraz próby automatycznej klasyfikacji. 
\subsubsection{Wybór klasyfikacji}
W obliczu takiej przeszkody postanowiono skupić się na wyborze mediów będących zakwalifikowanymi jako nierzetelne oraz przeciwstawić je do grupy mediów profesjonalnych oraz tych o dużej popularności w\,kontekście liczby odsłon serwisu. Dokonanie takiego podziału skupiającego się na oddzieleniu mediów nierzetelnych jest najbardziej rozsądnym podejściem, ponieważ łatwiej można stwierdzić, że jeśli coś spełnia wystarczającą liczbę przesłanek wskazujących na jego nierzetelność można je uznać za nierzetelne. Natomiast dużo trudniejsze jest określenie, że źródło jest wiarygodne. Dlatego warto przeciwstawić źródła nierzetelne do innych, które jako takie nie zostały zakwalifikowane. Dodatkowo zdecydowano się na wyodrębnienie stron zajmujących się sprawdzaniem faktów czyli tzw. Factchecking.
\subsubsection{Opis klas danych}
\par
Nierzetelne – lista mediów została zaczerpnięta z\,pracy „Junk news agregator” \cite{liotsiou2019junk} oraz „7 language survey” \cite{marchal2019junk} zespołu COMPROP. Tam nazywana jest jako Junk news. Składa się łącznie z\,13 domen internetowych, ponieważ w\,tych dwóch pracach listy pokrywają się. Aby domena znalazła się na takiej liście była oceniana przez trzech ekspertów według wcześniej określonych zasad. Metodę klasyfikacji mediów nierzetelnych przez zespół COMPROP przedstawiono dokładnie w\,rozdziale \ref{metodaocenyzrodel}.
\par
Mainstream – klasa mediów przeciwstawianych mediom nierzetelnym oznaczająca media profesjonalne oraz popularne wśród polskich użytkowników Internetu. Lista została stworzona na podstawie kilku źródeł naukowych przedstawionych poniżej. Praca „7 language survey” w\,której podano również media zakwalifikowane jako „junk” udostępniła również 5 domen uznanych przez twórców jako profesjonalne skąd zaczerpnięto je do tej pracy. Listę kolejnych serwisów pobrano z\,pracy Roberta Gorwy na temat botów w\,polskojęzycznej przestrzeni Twittera\cite{gorwa2017computational}. Listę dopełniono czołowymi serwisami informacyjnopublicystycznymi z\,badania Gemius przeprowadzonego w\,okresie 915 marca 2020\cite{GemiusSerwisy2020}. Łącznie otrzymano listę 16 domen, ponieważ część pokrywała się, ale również należało zignorować niektóre serwisy niepasujące do klasyfikacji takie jak serwis fakt.pl będący tabloidem i\,nie wpisujący się w\,tematykę pozostałych.
\par
Factcheck – klasa mediów zajmująca się sprawdzaniem prawdziwości wypowiedzi polityków oraz publikowanych plotek o tematyce politycznej. Pobrano listę pięciu takich serwisów opisanych w\,pracy przeprowadzonej przez NASK\cite{NASKZjawiskoDezinformacji2019}. 
