
\subsection{Wybór platformy social media}
Dwoma największymi mediami społecznościowymi w\,Polsce udostępniającymi funkcję dzielenia się informacjami tekstowymi przez użytkowników są Facebook oraz Twitter\cite{GemiusSerwisy2019}. Platformy te były już wykorzystywane w\,celu zbierania danych do badań, ponieważ są doskonałym źródłami, z\,których można uzyskać wiedzę na temat preferencji użytkowników. Między innymi z\,tych powodów postanowiono przyjrzeć się możliwościom jakie dają wystawione przez nie API, w\,celu uzyskania danych do badań przeprowadzanych w\,tej pracy.

\subsubsection{Facebook API}
Czytając prace badawcze na temat fałszywych informacji z\,ostatnich lat wiele z\,nich wykorzystuje jako źródło do pobrania danych służących analizie tego zagadnienia, platformę Facebook oraz należące do niej API\footnote{\url{https://developers.facebook.com}}. Było ono wygodnym narzędziem pozwalającym na pobranie informacji o publicznym koncie takie jak listy osób, które je ‘lubią’ lub obserwują. Istniała również możliwość pobierania publicznych postów opublikowanych na dowolnym publicznym koncie wraz z\,podłączonymi komentarzami, reakcjami oraz statystykami udostępnień. Użycie tych funkcji było udostępnione dla każdego uwierzytelnionego użytkownika, tak jak dostępne są dla niego w\,przypadku korzystania z\,platformy lub aplikacji mobilnej. Dzięki darmowemu oraz wszechstronnemu dostępowi do informacji publikowanych jako publiczne przez użytkowników Facebook API było wygodnym narzędziem służących do celów zbierania danych o informacjach publikowanych w\,mediach społecznościowych oraz ich popularności wśród odbiorców. Dzięki temu powstało kilka prac podejmujących próbę rozpoznania wielkości problemu jakim są fałszywe informacje w\,mediach społecznościowych oraz szukających rozwiązania na ich automatyczne rozpoznawanie. 
\par
Niestety wraz ze zmianami wprowadzonymi do Facebook API na koniec kwietnia 2019 roku został zamknięty dostęp do części usług\footnote{\url{https://developers.facebook.com/blog/post/2019/04/25/apiupdates/}}. Obecnie pobieranie informacji poprzez API możliwe wyłącznie dla kont których uwierzytelniony użytkownik jest właścicielem lub administratorem.
\subsubsection{Twitter API}
Platforma Twitter również posiada swoje API\footnote{\url{https://developer.twitter.com/en}}. Może ono służyć zarówno do łatwiejszego prowadzenia własnego konta, uprawiania marketingu lub zbierania danych do analizy. Istnieje kilka projektów oraz prac badawczych wykorzystujących dane pobrane przez Twitter API do badania problemu rozprzestrzeniania się nierzetelnych informacji w\,social mediach\cite{marchal2019junk}\cite{gorwa2017computational}\cite{vosoughi2018spread}. Twitter działa na metodzie mikroblogów co oznacza, że każdy użytkownik może publikować posty które inni użytkownicy udostępniają. Więcej o tej platformie jako środku do rozprzestrzeniania informacji opisano w\,rozdziale \ref{istniejacebadaniatwitter} oraz daje dostęp do interesujących mnie informacji.
\par
Powyższe api posiada jednak pewne ograniczenia. Pierwszym z\,nich jest konieczność posiadania konta na Twitterze dzięki któremu można utworzyć konto dla deweloperów. Każdy dostęp do API zabezpieczony jest poprzez standard autoryzujący OAuth. Aby uzyskać prywatny klucz należy wypełnić formularz informując jak i\,w jakim celu będziemy używać tego API oraz czy i\,w jaki sposób zamierzamy przechowywać, przetwarzać oraz publikować uzyskane dane.
Kolejne ograniczenie dotyczy dostępu do funkcjonalności. Twitter API posiada trzy poziomy, gdzie tylko standardowy jest darmowy. Kolejne poziomy, premium oraz enterprise, przeznaczone są dla firm gotowych zainwestować w\,bardziej zaawansowane rozwiązania. Poziom standard daje jednak dostęp do wystarczającej liczby funkcjonalności, aby umożliwić przeprowadzaną przeze mnie analizę. 
Trzecim, najbardziej uciążliwym ograniczeniem, są limity liczby możliwych zapytań lub dostarczonych danych. Większość endpointów ma nałożoną maksymalną liczbę wywołań którą można dokonać do nich w\,ciągu 15 minut używając jednego klucza autoryzującego. Innym rodzajem ograniczenia ilościowego zastosowanego na wybranych endpointach jest maksymalna liczba jednostek obiektów które możemy otrzymać w\,odpowiedzi. Przykładem takiego ograniczenia jest zapytanie o posty danego użytkownika, które zwraca maksymalnie 3200 ostatnio opublikowanych postów użytkownika.
\subsubsection{Użyte enpointy i ich ograniczenia}
W tabeli poniżej przedstawiono informacje o funkcjonalnościach Twitter API użytych do pobrania danych oraz informacje o ich ewentualnych ograniczeniach ilościowych.  

\begin{table}[!h] \label{tab:endointytwitter} \centering
\caption{Informacje o użytych enpointach Twitter API.}
\begin{tabular} { | m{4,5cm} | m{3,5cm}| m{4,5cm} | } \hline
Opis & Twitter Enpoint & Ograniczenia \\  \hline \hline
Pobierz informacje o\,użytkowniku & GET users/show & 900 zapytań / 15 minut \\ \hline
Pobierz listę identyfikatorów followersów \mbox{użytkownika} & GET followers/ids & 15 zapytań / 15 minut \\ \hline
Pobierz tweety użytkownika & GET statuses/ user\_timeline & Max 3200 najnowszych tweetów \mbox{100 000 zaptań /dzień} \\ \hline
Pobierz listę identyfikatorów retweeterów tweeta & GET statuses/ retweeters/ids & Max 100 identyfikatorów dla tweeta \mbox{75 zapytań / 15 min} 
 \\ \hline
\end{tabular}
\end{table}

Dla ułatwienia pracy przy korzystaniu z\,Twitter API użyto ogólnodostępną bibliotekę Twitter4J  stworzoną dla języka Java i\,objętą licencją Apache Licence 2.0. Biblioteka ta udostępnia klasy oraz metody wspomagające integrację nowej aplikacji z\,serwisem Twitter. Dzięki jej zaadoptowaniu w\,nowej aplikacji nie musiały znaleźć się bezpośrednie zapytania http co zwiększyło czytelność kodu. 

\subsection{Słownik wyrażeń}
Poniżej zostaną przedstawione wybrane słowa o szczególnym znaczeniu dla platformy Twitter. Większość z\,nich pochodzi z\,języka angielskiego lub jest neologizmami. Z\,tego powodu w\,tekście pracy będą używane również polskie synonimy tych słów. Dodatkowo ze względu na specyficzne ich znaczenie, zostanie przedstawiony ich opis.

\begin{table}[!h] \label{tab:slowniktwitter} \centering
\caption{Słownik domenowy portalu Twitter.}
\begin{tabular} { | m{2cm} | m{4,5cm}| m{7cm} | } \hline
Słowo & Używane synonimy & Opis \\  \hline \hline
Konto & Użytkownik, osoba,
  profil. Tutaj też: media, serwis & Konto
  znajdujące się na platformie tweeter, może należeć do prywatnego użytkownika
  lub do serwisu internetowego. Zawiera informacje o sobie, swoje tweety oraz
  retweety. \\ 
\hline
Follower & Subskrybent, śledzący. & Użytkownik który
  wyraził chęć otrzymywania tweetów konta śledzonego na swojej głównej stronie
  przeglądania \\ 
\hline
Tweet & Post, status,
  treść. & Krótka wiadomość
  tekstowa i/lub zawierająca inną formę medialną wyrażająca informację lub
  opinię użytkownika, który go publikuje. \\ 
\hline
Retweet & Udostępnienie,
  \mbox{podanie dalej.} & Opublikowanie postu
  stworzonego przez innego użytkownika na swoim koncie. Ma na celu przedstawienie
  swojej opinii w\textbackslash{},stosunku do jego treści lub chęć rozpowszechnienia
  udostępnianego tweeta. Używane jako czasownik oraz rzeczownik. \\ 
\hline
Retweeter & Udostępniający. & Użytkownik
  który retweetuje tweet. \\
\hline
\end{tabular}
\end{table}