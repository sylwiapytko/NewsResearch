\subsection{Słownik wyrażeń}
Poniżej zostaną przedstawione wybrane słowa o szczególnym znaczeniu dla platformy Twitter. Większość z\,nich pochodzi z\,języka angielskiego lub jest neologizmami. Z\,tego powodu w\,tekście pracy będą używane również polskie synonimy tych słów. Dodatkowo ze względu na specyficzne ich znaczenie, zostanie przedstawiony ich opis.

\begin{table}[!h] \label{tab:slowniktwitter} \centering
\caption{Słownik domenowy portalu Twitter.}
\begin{tabular} { | m{2cm} | m{4,5cm}| m{7cm} | } \hline
Słowo & Używane synonimy & Opis \\  \hline \hline
Konto & Użytkownik, osoba,
  profil. Tutaj też: media, serwis, źródło. & Konto
  znajdujące się na platformie tweeter, może należeć do prywatnego użytkownika
  lub do serwisu internetowego. Zawiera informacje o sobie, swoje tweety oraz
  retweety. \\ 
\hline
Follower & Subskrybent, śledzący. & Użytkownik który
  wyraził chęć otrzymywania tweetów konta śledzonego na swojej głównej stronie
  przeglądania. \\ 
\hline
Tweet & Post, status,
  treść. & Krótka wiadomość
  tekstowa i/lub zawierająca inną formę medialną wyrażająca informację lub
  opinię użytkownika, który go publikuje. \\ 
\hline
Retweet & Udostępnienie,
  \mbox{podanie dalej.} & Opublikowanie postu
  stworzonego przez innego użytkownika na swoim koncie. Ma na celu przedstawienie
  swojej opinii w\,stosunku do jego treści lub chęć rozpowszechnienia
  udostępnianego tweeta. Używane jako czasownik oraz rzeczownik. \\ 
\hline
Retweeter & Udostępniający. & Użytkownik
  który retweetuje tweet. \\
\hline
\end{tabular}
\end{table}