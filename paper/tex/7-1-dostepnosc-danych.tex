\subsection{Dostępność danych oraz wstępna analiza}
Poniżej zostaną dokładniej przedstawione typy pobranych danych oraz ograniczenia z jakimi należało się zmierzyć. Z tego też powodu znajdzie się tutaj również wstępna analiza danych w szczególności ich liczby oraz dat z jakich pochodzą. 
\subsubsection{Dostępność kont }
Na podstawie przygotowanych wcześniej list serwisów internetowych dopasowanych do odpowiedniej kategorii, co opisano w\,rozdziale \ref{wybor-danych}, przeprowadzono proces dopasowywania ich oficjalnych kont na platformie Twitter. Aby konto zostało uznane za oficjalne musiał istnieć do niego odnośnik na stronie serwisu lub w\,opisie konta musiał znajdować się link do serwisu. 
\par
W większości udało się dopasować odpowiednie konta. W\,niektórych przypadkach nie było to jednak możliwe. W\,większości takich przypadków, nie udało się znaleźć odpowiedniego konta na platformie Twitter które można by uznać za oficjalne. Było też kilka przypadków, że konto wspomniane w\,badaniach, z\,których pobrano o nim informacje już nie istniało lub zostało zawieszone.
\par
Dla niektórych z\,serwisów zdecydowano się o wybranie jednego z\,oficjalnych kont, takiego które bardziej spełnia tematykę informacyjną. Do tych przypadków należą duże serwisy takie jak Onet, Interia i\,RadioZET, które publikują treści o szerokiej gamie tematycznej, od sportu i\,rozrywki po informacje o pogodzie.  W\,tych przypadkach wybrano to z\,ich oficjalnych kont którego główną zawartością są teksty o tematyce typowo informacyjnej. 
\par
Dodatkowo w\,dwóch przypadkach zdecydowano się nie ograniczać do przydzielenia jednego konta. Tymi przypadkami są portale stacji telewizyjnych TVP oraz TVN. Dla obu z\,nich zdecydowano o pobraniu zarówno oficjalnego konta należącego do tych grup kanału informacyjnego, jak również pobrano oficjalne konta ich głównych programów informacyjnych. 
Łącznie udało się znaleźć 10 kont należących do klasy nierzetelne, 14 należące do mainstream oraz 5 factcheck.

\begin{table}[!h] \label{tab:kontatwitter} \centering
\caption{Lista dostępnych kont z\,platformy Twitter z\,podziałem na klasy.}
\begin{tabular}{|m{3,4cm}| m{3,4cm} | m{3,4cm}| m{3,4cm} |} 
\hline
 NIERZETELNE\textit{}  &  MAINSTREAM \textless{}0,5 mln \textit{}  &  MAINSTREAM \textgreater{}0,5mln\textit{}  &  FACTCHECK\textit{}  \\ 
\hline \hline
 Dlapolski\textit{}  &  natematpl\textit{}  &  tvn24\textit{}  &  demaskator24\textit{}  \\ 
\hline
 Matka\_Kurka\textit{}  &  wgospodarce\textit{}  &  gazeta\_wyborcza\textit{}  &  oko\_press\textit{}  \\ 
\hline
 RepublikaTV\textit{}  &  KRESYPL\textit{}  &  gazetapl\_news\textit{}  &  konkret24\textit{}  \\ 
\hline
 PikioPL\textit{}  &  bankier\_pl\textit{}  &  tvp\_info\textit{}  &  DemagogPL\textit{}  \\ 
\hline
 Niezaleznapl\textit{}  &  WiadomosciTVP\textit{}  &  ~\textit{}  &  AntyFakePL\textit{}  \\ 
\hline
 wSensie\textit{}  &  FaktyTVN\textit{}  &  ~\textit{}  &  ~\textit{}  \\ 
\hline
 MediaNarodoweMN\textit{}  &  rzeczpospolita\textit{}  &  ~\textit{}  &  ~\textit{}  \\ 
\hline
 wPrawopl\textit{}  &  OnetWiadomosci\textit{}  &  ~\textit{}  &  ~\textit{}  \\ 
\hline
 Medialne\textit{}  &  Interia\_Fakty\textit{}  &  ~\textit{}  &  ~\textit{}  \\ 
\hline
 CrowdMedia\_PL\textit{}  &  RadioZET\_NEWS\textit{}  &  ~\textit{}  &  ~\textit{}  \\
\hline
\end{tabular}
\end{table}


\par
W trakcie procesu poszukiwania odpowiednich kont serwisów zauważono, że konta zakwalifikowane do kategorii Mainstream znacznie różnią się od siebie w\,popularności liczonej jako liczba followersów. Podjęto więc decyzję o wydzieleniu w\,śród nich podgrupy. Podział oparto na liczbie follwersów ponieważ najpardziej popularne konto posiadało 1,4 miliona followersów, gdy inne konta posiadają ich kilka lub kilkanaście tysięcy. Eksperymentalnie granicę postawiono na liczbie pół miliona followersów, ponieważ celem było wydzielenie tych najbardziej popularnych kont oraz w\,dalszych krokach zbadanie ich rzeczywistego wpływu na ilość udostępnianych treści. Zbiór ten będzie czasami nazywany jako Bigmainstream.

