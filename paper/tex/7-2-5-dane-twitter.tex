\subsection{Dane o followersach}
W celu dokonania analizy grup odbiorców odpowiednich źródeł pobrano informację o użytkownikach będących „followersami” interesujących nas kont. Bycie followersem w\,praktyce oznacza subskrypcję, czyli tak zwane śledzenie. Dzięki takiemu zabiegowi posty publikowane na koncie śledzonym będą zawsze pojawiały się na głównej stronie użytkownika, który włączył subskrypcję. Dzięki temu liczba followersów jest dobrym wyznacznikiem popularności konta, oznacza ona bowiem jak dużo użytkowników chce czytać nowe publikacje tego konta. 
\par
W przypadku pobierania informacji o followersach, oprócz całkowitej sumy, pobierano jedynie ich identyfikator numeryczny. Jest to dobre rozwiązanie, ponieważ zachowujemy anonimowość użytkowników innych niż analizowani. Poważnym problemem okazały się ograniczenia ilościowe Twitter API. Powodują one, że w\,odpowiedzi na jedno zapytanie możliwe jest otrzymanie maksymalnie 5000 użytkowników a\,w czasie 15 min można wysłać jedynie 15 zapytań. Z\,tego powodu listę followersów zebrano tylko dla kont, które posiadają więcej niż 500 tysięcy followersów. Dla kont, które miały więcej stworzono osobną klasę nazwaną „Bigmainstream”. Są to bardzo popularne konta posiadające od pół do półtora miliona followersów. 
\subsection{Zebrane tweety}
Mimo, że niektóre konta posiadają nawet ponad ćwierć miliona opublikowanych tweetów ograniczenia jakie nakłada API pozwoliły na pobranie jedynie 3200 ostatnio opublikowanych postów. Aby zapewnić spójność zebranych danych wszystkie dane zostały zebrane w\,przedziale czasu od 02.04.2020 do 13.04.2020 w\,trzech partiach. Dla większości kont pobrano maksymalną możliwą liczbę tweetów i\,łącznie zebrano 107047 postów z\,dwudziestu dziewięciu kont. 
\par
Uwzględniając ograniczenie jakim jest możliwość pobrania jedynie ograniczonej liczby ostatnich tweetów użytkownika oraz różną częstotliwość z\,jaką nowe treści są publikowane na kątach, sprawdzono datę najstarszych pobranych postów dla każdego z\,analizowanych użytkowników. Najstarsze zebrane tweety pochodzą z\,2016 i\,2017 roku.  Natomiast dla konta o największej częstotliwości publikowania postów w\,ostatnim czasie ostatni tweet jaki był możliwy do pobrania pochodzi z\,13.02.2020. Z\,tego powodu ograniczono tweety poddane dalszej analizie do tych opublikowanych pomiędzy 13.02.2020 a\,13.04.2020. Daje to 31 dni z\,których łącznie pobrano 36672 tweetów. Co ważne, należy zwrócić uwagę, że nie są to jedynie posty oryginalnie utworzone przez użytkownika, ale również wszystkie, które zdecydował się opublikować na swojej tablicy poprzez udostępnienie postu utworzonego przez innego użytkownika. 

\begin{table}[!h] \centering
\caption{Liczba zebranych tweetów opublikowanych w\,okresie pomiędzy 13.02.2020 a\,12.04.2020 dla każdej z\,klas.}  \label{tab:zebranetweety}
\begin{tabular}{|l|r|} 
\hline
~Klasa & Zebrane tweety \\ 
\hline \hline
NIERZETELNE & 10466 \\ 
\hline
MAINSTREAM \textless{} 0,5 mln & 17484 \\ 
\hline
MAINSTREAM \textgreater{} 0,5mln & 8078 \\ 
\hline
FACTCHECK & 644 \\
\hline 
Suma & 36672 \\
\hline 
\end{tabular}

\end{table}

\subsection{Dane o użytkownikach retweetujących tweety}
Ze względu na ograniczenia jakie posiada API Twitter, spowalniające pobieranie informacji o retweetach do 3 zapytań na minutę, pobierano takie informacje dla maksymalnie 1800 tweetów na dobę. Z\,powodu tej niekorzystnej sytuacji, chcąc zachować spójność danych, pobierano dane z\,kolejnych dni wstecz, rozpoczynając od 10tego kwietnia. Dzięki temu, że ogólna informacja o ilości udostępnień tweeta była dostępna wraz z\,innymi, pobranymi wcześniej, danymi, to można było maksymalnie ograniczyć ilość zapytań, aby uzyskać potrzebne dane. Przy wykonywaniu zapytań do Twitter API o użytkowników udostępniających post brano pod uwagę tylko te tweety, o których posiadano informacje, że zostały udostępnione przynajmniej raz. Zakończono pobieranie informacji o udostępnieniach, gdy posiadano je dla 15 tysięcy oryginalnych tweetów. To oznacza, że w\,zbiorze zebranych danych są one dostępne dla tweetów opublikowanych pomiędzy 20.03 a\,10.04.2020r włączając te daty.  