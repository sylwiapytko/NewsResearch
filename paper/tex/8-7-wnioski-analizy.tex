\subsection{Wnioski z analizy}
Celem przeprowadzonej analizy było zbadanie charakterystyk kont, treści przez nie publikowanych oraz ich odbiorców. W części badań skupiano się na właściwościach wszystkich kont w danej klasie aby przeprowadzić całościową analizę porównawczą charakterystyk przedstawianych klas. Porównywano trzy klasy: konta nierzetelne, mainstream oraz factcheck. Z powodu dużej różnicy w ilości tzw. followersów, czyli odbiorców kont mainstream zdecydowano wyróżnić w śród tych kont dwa niezależne niepokrywające się podzbiory. Granicę ustalono na 0,5 mln followersów. Okazało się być to dobrym wyborem ponieważ kolejna największa liczba followersów to 0,2 mln.  
\par Różnicę w ilości odbiorców widać patrząc na średnią liczbę udostępnień postów przez użytkowników. Konta mainstream o dużej liczbie followersów posiadają średnio więcej udostępnień, niż te z mniejszą liczbą. Różnicę widać również w liczbie publikowanych nowych postów. Liczba postów oraz followersów niekoniecznie jednak wpływa bezpośrednio na liczbę udostępnień. Konta należące do serwisów factcheck mimo posiadania stosunkowo małej liczby followersów otrzymują średnio najwięcej udostępnień ze wszystkich grup.
\par
Zbadano również zachowania użytkowników udostępniających posty. Okazuje się, że prawie połowa zebranych użytkowników udostępniła tylko i wyłącznie jeden post. Najbardziej aktywnymi, pod względem liczby udostępnień, użytkownikami są użytkownicy udostępniający posty kont kategorii nierzetelne.
\par
Dodatkowo zbadano połączenia pomiędzy kontami, w formie wzajemnych udostępnień swoich postów. Chciano w ten sposób sprawdzić czy istnieją bezpośrednie lub niebezpośrednie zależności między nimi w utworzonych klasach lub pomiędzy nimi. Okazuje się jednak, że takie połączenia istnieją jedynie pomiędzy kontami należącymi do tego samego wydawcy. Oznacza to, że badane konta zazwyczaj nie udostępniają informacji pochodzących z innych źródeł. Największy procent oryginalnych treści posiadają konta z\,kategorii nierzetelne.  Zbadano również, jakie domeny są linkowane w postach badanych kont, i podobnie okazuje się, że konta linkują informacje jedynie z serwisów do których należą. 
\par
Po tej analizie widać, że najbardziej obiecującą właściwością charakteryzującą konta w poszczególnych klasach mogą być użytkownicy którzy udostępniają ich posty. Mimo, że połowa użytkowników udostępnia około 2 posty to istnieje  nadal niemała grupa osób która udostępnia ich bardzo dużo. Na tej podstawie można próbować klasyfikacji postów. Należy zobaczyć jak duża grupa osób jest spolaryzowana na konkretną klasą postów, oraz jak dużo osób udostępnia posty pochodzące z kont zaklasyfikowanych do różnych klas. Mając te informacje można potraktować użytkowników udostępniających posty jako charakterystykę kont. Można badać prawdopodobieństwo preferencji użytkowników lub posiadając wiedzę o użytkownikach przewidywać prawdopodobieństwo przynależności informacji do konkretnej klasy. 