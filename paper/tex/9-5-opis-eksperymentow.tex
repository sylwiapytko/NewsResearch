\subsection{Opis przeprowadzonych badań}
W poniższym podrozdziale zostaną przedstawione i opisane zaplanowane badania jakie będą przeprowadzone w kolejnym podrozdziale. Ma to na celu przybliżenie metodyki badań, sposobu przeprowadzenia ich oraz opis w jaki sposób pozwolą one osiągnąć odpowiedzi na zaplanowane wcześniej cele.
\subsubsection{Precyzja modelu}
Na obu przygotowanych zbiorach przeprowadzono kilka eksperymentów. Głównym celem było sprawdzenie z\,jaką dokładnością można klasyfikować tego typu dane oraz który z\,algorytmów daje najlepsze wyniki dla określonych zbiorów. Do tego celu zostanie wykorzystana 5-krotna walidacja krzyżowa (ang. 5-fold validation) polegająca na pięciokrotnym sprawdzeniu modelu przy zbiorze uczącym składającym się z\,80\% losowo wybranych danych. 
\subsubsection{Wielkość zbioru uczącego a\,precyzja}
Celem drugorzędnym było sprawdzenie jak duży musi być zbiór uczący, aby otrzymać zadowalające wyniki. W\,przypadku zastosowania automatycznej klasyfikacji postów w\,rzeczywistym systemie jest to bardzo ważna kwestia. Odpowiednie zaklasyfikowanie postów jest zadaniem bardzo kosztownym. Chcąc wykorzystać automatyczną klasyfikację poza środowiskiem badawczym należy wiedzieć czy można takie rozwiązanie skalować na zbiory testowe dużo większe od zbiorów uczących. 
\par
Dla każdej z\,wielkości zbiorów uczących zostanie dokonane 50 powtórzeń algorytmu z\,różnymi danymi w\,zbiorze uczącym. Jako precyzja zostanie pokazana średnia tych 50 powtórzeń. Natomiast prezentowane odchylenie standardowe będzie dotyczyło różnic w\,precyzji pomiędzy przeprowadzonymi powtórzeniami. Zastosowano takie podejście, ponieważ skuteczność nauki modelu klasyfikacji w\,głównej mierze zależy od dostosowania jego zbioru uczącego do zbioru testowego. Mogą się zdarzyć przypadki bardzo o\,bardzo wysokiej precyzji, ale nie świadczy to o\,sukcesie modelu, jeśli pozostałe przypadki rozłożenia danych będą otrzymywały bardzo niskie wyniki.
\par
Podział zbioru na część uczącą i\,testową będzie dokonany semi-logarytmicznie. Zostanie zbadane osiem różnych wielkości zbioru uczącego.  Podział zbioru dokonywany jest w\,taki sposób, że określona część danych zostanie potraktowana jako zbiór uczący, natomiast cała pozostała część zbioru będzie służyła zbadaniu precyzji nauczonego modelu jako zbiór testowy. Celem jest zobaczenie, jak mała część zbioru poddanego klasyfikacji musi mieć znane etykiety, aby automatyczna klasyfikacja miała sens, dając zadowalającą precyzję klasyfikacji nieznanych danych.
\subsubsection{Transmisja wiedzy w\,modelu}
Z tym związana jest również kwestia przenoszenia wiedzy uzyskanej o\,grupie postów na posty pochodzące z\,innych kont. Należy sprawdzić czy, dzięki udostępnieniom użytkowników, system klasyfikacji jest w\,stanie poprawnie nadać etykiety postom z\,innego konta niż zawarte w\,zbiorze uczącym. W\,tym celu zostaną przeprowadzone dwa eksperymenty. W\,pierwszym z\,nich do zbioru uczącego zostaną podane wszystkie posty oprócz postów należących do jednego konta. Taki test zostanie przeprowadzony dla wszystkich możliwych kont. Oraz w\,drugim eksperymencie do zbioru uczącego zostanie podana tylko połowa kont, a\,druga połowa pozostanie w\,zbiorze testowym i\,tak dla wszystkich możliwych kombinacji.