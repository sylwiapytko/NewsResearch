\subsection{Wnioski z badań}
Na podstawie wykonanych badań można jasno określić, że najlepszym sposobem do klasyfikacji tego typu danych okazała się regresja logistyczna. Uzyskane dzięki niej wyniki przewyższały wyniki algorytmu harmonicznego zarówno dla zbioru pełnego jak i\,dla zbioru intersekcji. Dodatkowo zbadano, że przy zbiorze pełnym wystarczy, aby znano klasyfikację jedynie dziesięciu procent badanych postów, aby z\,precyzją 90\% określić klasy pozostałych postów. 
\par
	Niestety jednak okazało się, że żaden z\,powyższych algorytmów nie spełnia dobrze zadania transmisji wiedzy w\,modelu, jeśliby odrzucić posty opublikowane przez konkretne konta do zbioru testowego. Nie jest to jednak wina wybranych algorytmów, ale ułożenia danych. Dla niektórych z\,kont klasyfikacja ich postów była dokonywana z\,prawie 100\% precyzją. Średni wynik zaniżały jednak inne konta, które nie miały wystarczająco wspólnych udostępniających użytkowników. 
\par
Dodatkowo warto zaznaczyć, że wszystkie powyżej opisane badania były robione kilkakrotnie dla różnej liczby pobranych postów rozpoczynając od 8 tysięcy postów. Przy zwiększaniu ilości danych nie zauważono znacznej poprawy wyników klasyfikacji dla wszystkich przedziałów podziału zbioru na uczący i\,testowy. Dlatego pozostano przy tweetach pobranych z\,okresu 20 dni. Po przeprowadzonej analizie badań można zauważyć, że wysoka precyzja utrzymuje się nawet dla 10\% danych w\,zbiorze uczącym, czyli wystarczy około 1200-1500 postów w\,zależności od zbioru.
