\newpage % Rozdziały zaczynamy od nowej strony.
\section{Praefatio}
\lipsum[1] \cite{goossens93}
\begin{figure}[!h]
    \label{fig:tradycyjne-logo-pw}
    \centering \includegraphics[width=0.5\linewidth]{logopw.png}
    \caption{Tradycyjne godło Politechniki Warszawskiej}
\end{figure}
\lipsum[2-3]
\begin{figure}[!h]
	\label{fig:nowe-logo-pw}
	\centering \includegraphics[width=0.5\linewidth]{logopw2.png}
	\caption{Współczesne logo Politechniki Warszawskiej}
\end{figure}
\lipsum[4-6]
Każdy post posiada zbiór cech, który może przyjmować jedną z dwóch możliwych wartości – 1, gdy dany użytkownik udostępnił dany post lub 0, gdy dany użytkownik nie udostępnił danego postu.

$$\forall\ i\in I\ ,\ u\in U\ \ \ \exists\ x_{iu}=f\left(i,u\right)$$

$$f\left(i,u\right)\ =\ 1,gdy użytkownik u udostępnił post i  \\       0,gdy użytkownik u nie udostępnił postu i $$

	Regresja logistyczna polega na wyuczeniu modelu poprzez znalezienia odpowiednich wag dla cech modelu.  W przedstawianym modelu są to wagi użytkowników w_u. W praktyce waga większa od zera oznacza, że użytkownik udostępnia w większości posty mainstreamowe a waga mniejsza od zera oznacza, że użytkownik udostępnia w większości posty należące do kategorii Junk. Na podstawie wyuczonych wag cech model oblicza prawdopodobieństwo p_i, że post należy do klasy mainstream.
\forall\ i\in I\ \ \ \exists\ p_i=\ \frac{1}{1+e^{-y_i}}\ \bigmy_i=\ \sum_{u\inU}{x_{iu}w_u}

